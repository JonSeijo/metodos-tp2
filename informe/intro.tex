\section{Introducción}

Este trabajo consiste en el desarrollo de una herramienta de OCR \textit{(Optical Character Recognition)} para el reconocimiento de dígitos manuscritos a partir de imágenes. Estos dígitos se encontrarán aislados y estarán entre 0 y 9. \\

Desarrollaremos algoritmos para realizar una clasificación en clases a partir de una base de datos etiquetada. Utilizaremos el método de kNN \textit{(k Nearest Neighbours)} y luego lo combinaremos con el método de PCA \textit{(Principal Component Analysis)}. \\

Por medio de experimentación intentaremos encontrar los mejores parámetros posibles, y para evitar problemas como \textit{overfitting} utilizaremos la técnica de \textit{K-fold cross validation}. \\

Para finalizar, veremos cómo se comporta nuestro clasificador para imágenes que no están rotuladas. En particular, consideraremos la base de datos de test de la competencia de \textit{Kaggle} en reconocimiento de dígitos y veremos cuáles fueron los resultados obtenidos. \\
