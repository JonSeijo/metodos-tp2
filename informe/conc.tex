\section{Conclusión}

En este trabajo analizamos e implementamos un sistema de reconocimiento de dígitos manuscritos a partir de imágenes. En primer lugar utilizamos una idea sencilla, kNN, que funcionó de una manera mucho mejor de lo esperada. Lidiando con problemas en los tiempo de ejecución y con la gran cantidad de dimensiones, desarrollamos el método PCA para reducir la cantidad de variables de las muestras. \\

Analizamos, con la técnica de K-fold Cross Validation, los resultados obtenidos y los niveles de Precision, Recall y F1 que se lograron al ir variando los diferentes parámetros de nuestros métodos. Con estos datos, estimamos cuáles eran aquellos parámetros que conseguían los mejores resultados posibles. \\

Medimos la eficiencia de ambos métodos, tanto en efectividad como en costo temporal, y encontramos que PCA es el mejor de ambos. \\

Utilizamos nuestro sistema para participar en la competencia de Kaggle, que utiliza una base de datos para la cuál no conocemos sus verdaderos valores, y así poder evaluar la efectividad de nuestros métodos. Dado que obtuvimos un puntaje máximo de 0.97542, concluimos que nuestro desarrollo sobre las técnicas de reconocimiento de dígitos fue satisfactorio. \\

% Esta lindo pero me suena muy forzado :p - jonno
% Comprendimos lo fundamental que resulta el estudio de los elementos teóricos vistos en la materia (en particular todo lo relacionado a autovalores, autovectores y diagonalización de matrices, aplicado en PCA) y desarrollamos satisfactoriamente la técnica de reconocimiento de dígitos.