\section{Conclusión}

Concluimos que quedó re zarpado en piola.\\



ESTO DE ABAJO ESTÁ COMO COSA PREELIMINAR, PUEDEN CAMBIAR, BORRAR Y AGREGAR LO QUE QUIERAN\\


En este trabajo analizamos e implementamos una técnica de reonocimiento de dígitos. Utilizamos el método PCA para reducir la cantidad de variables de las muestras y agilizar la ejecución y la técnica kNN en el entrenamiento del programa. Luego corroboramos con K-fold los niveles de precision, recall y f1 obtenidos en dicho entrenamiento.\\

Estimamos parámetros de las técnicas utilizadas (PCA, kNN y Kfold) para buscar los mejores resultados y que consuman menor tiempo de cómputo.\\

Finalmente utilizamos la implementación realizada para participar en la competencia de Kaggle y evaluar la efectividad de nuestro programa.\\

Comprendimos lo fundamental que resulta el estudio de los elementos teóricos vistos en la materia (en particular todo lo relacionado a autovalores, autovectores y diagonalización de matrices, aplicado en PCA) y desarrollamos satisfactoriamente la técnica de reconocimiento de dígitos.