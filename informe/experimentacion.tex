\section{Resultados finales y experimentación}
\subsection{Resultados}

\todo[inline]{Experimentacion con knn variando los k-knn y los K-folds, para encontrar el mejor parametro posible}

\todo[inline]{Experimentacion de tiempo variando la cantidad de imagenes, no hace falta mil experimentos, un par de puntos clave, digamos 100, 1000, 2000, 5000, 10000, 20000, 40000}

\todo[inline]{Comparacion tiempos con, digamos, 10000 training entre knn a secas y con psa.}

\todo[inline]{Comparacion accuaracy y cosas con, digamos, 10000 training entre knn a secas y con psa.}

\todo[inline]{Podemos variar los K-kfolds si la base de entrenamiento es 'chica'}

\todo[inline]{Para entrenamientos con TODAS las imagenes y PSA, QUIZA nos convenga fijar los kfold en, nose, 5, y solo variar los demas parametros. Es RE costoso, (horas) calcular los K folds con un training grande (CON PSA).  Esta piola la idea de agustin de guardarnos en un txt la matriz de covarianza. \\
En limpio, la idea seria: Para 5 folds calcular y guardar las 5 matrices de covarianza, y despues experimentar variando los alphas y los k-knn. Esto seria 'barato' dentro de todo y podrian hacerce un par de graficos. Si hay tiempo variar el k-KFOLD pero lo dejaria para lo ultimo, mejor conseguir resultados concretos antes.}

\todo[inline]{Al final, mostrar resultados de Kaggle con y sin PSA, ir probando con un par de buenos parametros cuando los obtengamos}

\todo[inline]{Me parece muy piola de probar con manuscritos hechos por nosotros. Solo si sobra tiempo!}