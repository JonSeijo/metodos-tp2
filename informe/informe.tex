\documentclass[a4paper]{article}
\usepackage[spanish]{babel}
\usepackage[utf8]{inputenc}
\usepackage{graphicx}
\usepackage{enumerate}
\usepackage{listings}
\usepackage{color}
\usepackage{indentfirst}
\usepackage{fancyhdr}
\usepackage{latexsym}
\usepackage{blkarray}
\usepackage{multirow}
\usepackage[colorlinks=true, linkcolor=black]{hyperref}
%\usepackage{makeidx}
%\usepackage{float}
\usepackage{calc}
\usepackage{amsmath, amsthm, amssymb}
\usepackage{amsfonts}
%\lstset{language=C}
\definecolor{gray}{gray}{0.5}
\definecolor{light-gray}{gray}{0.95}
\definecolor{orange}{rgb}{1,0.5,0}

\input{informe/page.layout}
% \setcounter{secnumdepth}{2}
\usepackage{underscore}
\usepackage{caratula}
\usepackage{url}
\usepackage{float}

\usepackage{commath}
\usepackage[draft]{todonotes}
\usepackage{caption}

\usepackage{algorithm}
\usepackage[noend]{algpseudocode}
\usepackage{array}
\usepackage{xcolor,colortbl}
\usepackage{amsthm}
\usepackage{mathtools}
\usepackage{listings}
\usepackage{svg}
\usepackage{systeme}
\usepackage{color,soul}
\usepackage{multirow}

%\captionsetup[figure]{labelformat=empty}% redefines the caption setup of the figures environment in the beamer class.

\makeatletter
\renewcommand*\env@matrix[1][*\c@MaxMatrixCols c]{%
 \hskip -\arraycolsep
 \let\@ifnextchar\new@ifnextchar
 \array{#1}}
\makeatother


\newcommand{\cod}[1]{{\tt #1}}
\newcommand{\negro}[1]{{\bf #1}}
\newcommand{\ital}[1]{{\em #1}}
\newcommand{\may}[1]{{\sc #1}}
\newcommand{\tab}{\hspace*{2em}}

\hypersetup{
 pdfstartview= {FitH \hypercalcbp{\paperheight-\topmargin-1in-\headheight}},
 pdfauthor={Grupo},
 pdfsubject={Dise\~{n}o}
}

\lstdefinestyle{customc}{
  backgroundcolor=\color{light-gray},
  belowcaptionskip=1\baselineskip,
  breaklines=true,
  numbers=left,
  xleftmargin=\parindent,
  language=C,
  showstringspaces=false,
  basicstyle=\footnotesize\ttfamily,
  keywordstyle=\bfseries\color{blue},
  commentstyle=\itshape\color{gray},
  identifierstyle=\color{black},
  stringstyle=\color{orange},
}

\lstdefinestyle{customasm}{
  backgroundcolor=\color{light-gray},
  belowcaptionskip=1\baselineskip,
  numbers=left,
  xleftmargin=\parindent,
  language=[x86masm]Assembler,
  keywordstyle=\bfseries\color{blue},
  basicstyle=\footnotesize\ttfamily,
  commentstyle=\itshape\color{gray},
}

\lstset{escapechar=@}


\begin{document}

\thispagestyle{empty}
\materia{Métodos Númericos}
\submateria{2do Cuatrimestre - 2017}
\titulo{Trabajo Práctico II}
% \subtitulo{Subtitulo}
\integrante{Jonathan Seijo}{592/15}{jon.seijo@gmail.com}
\integrante{Lucas De Bortoli}{736/15}{lu_cas_.97@hotmail.com.ar}
\integrante{Roberto Grings}{345/08}{robertoegrings@gmail.com}
\integrante{Agustín Penas}{668/14}{agustinpenas@gmail.com}

\makeatletter

\maketitle
\newpage

\thispagestyle{empty}
\vfill

\thispagestyle{empty}
\vspace{3cm}
\tableofcontents
\newpage

\newenvironment{myindentpar}[1]
{\begin{list}{1}
         {\setlength{\leftmargin}{#1}}
         \item[]
}
{\end{list} }

%\normalsize
\newpage

% -------------------------------------------------------
% Introducción Teórica
% -------------------------------------------------------
\section{Introducción}

Este trabajo consiste en el desarrollo de una herramienta de OCR \textit{(Optical Character Recognition)} para el reconocimiento de dígitos manuscritos a partir de imágenes. Estos dígitos se encontrarán aislados y estarán entre 0 y 9. \\

Desarrollaremos algoritmos para realizar una clasificación en clases a partir de una base de datos etiquetada. Utilizaremos el método de kNN \textit{(k Nearest Neighbours)} y luego lo combinaremos con el método de PSA \textit{(Principal Component Analysis)}. \\

Por medio de experimentación intentaremos encontrar los mejores parámetros posibles, y para evitar problemas como \textit{overfitting} utilizaremos la técnica de \textit{K-fold cross validation}. \\

Para finalizar, veremos cómo se comporta nuestro clasificador para imágenes que no están rotuladas. En particular, consideraremos la base de datos de test de la competencia de \textit{Kaggle} en reconocimiento de dígitos y veremos cuáles fueron los resultados obtenidos. \\



% -------------------------------------------------------
% Desarrollo
% -------------------------------------------------------
\newpage
\section{kNN}

\todo[inline]{Como esta representada nuestra informacion}

\todo[inline]{Que es knn}

\todo[inline]{Por que nos sirve en nuestro problema? como lo usamos?}

\todo[inline]{Usar alguna imagen ilustrativa}

\todo[inline]{Implementacion, quiza hablar sobre optimizaciones y estructuras, minheap y eso}

\todo[inline]{Cuales son los posibles problemas que traeria? Dimensionalidad. Aclarar que corroboraremos esto en la seccion de experimentacion}


\section{PSA}

\todo[inline]{Que es psa}

\todo[inline]{Por que usariamos psa en nuestro problema}

\todo[inline]{Explicar con detalle el metodo}

\todo[inline]{Mencionar que necesitamos los autovec pero explicar como en la seccion siguiente}

\todo[inline]{Posibles problemas? Muy costoso crear la matriz de covarianza si tenemos una muestra muy grande}



\section{Deflación y método de la potencia}

Cómo habíamos mencionado, para poder diagonalizar la matriz de covarianza necesitamos conseguir los autovalores y autovectores. Conseguir los autovalores de la manera tradicional implica encontrar las raíces de un polinomio de grado 784, y no es algo que estemos dispuestos a intentar. Utilizaremos entonces el método iterativo conocido como \textit{Método de la potencia}.
\begin{algorithm}
    \caption{Método de la potencia (Matriz $B$, Vector $x_0$, Int $iters$)}
    \begin{algorithmic}[h]
        \State{$v \gets x_0$}
        \For{$i \gets 1$ to $iters$}
            \State{$v \gets \frac{Bv}{\norm{Bv}}$} \\
        \EndFor
        \State  $\lambda \gets \frac{v^tBv}{v^tv}$
        \State return $(\lambda, v)$

    \end{algorithmic}
\end{algorithm}

Podemos estar seguros de que los autovalores son números reales positivos, pues la matriz de covarianza es una matriz simétrica. Además, también por ser simétrica sabemos que vamos a poder conseguir una base de autovectores ortonormales, que es exactamente lo que necesitamos. \\

Con el método de la potencia conseguimos únicamente el autovalor dominante (y un autovector correspondiente). Como queremos conseguir una base, extendemos el método de la potencia utilizando el método de deflación. Esto funciona por la siguiente observación:

$$ B - \lambda v_1 {v_1}^{t} \text{\hspace{1cm}tiene los mismos autovalores que B (excepto $\lambda$).} $$

Por lo tanto, cuando apliquemos el método de la potencia en la matriz resultante, el autovalor que conseguiremos será distinto del $\lambda$ obtenido inicialmente. \\

Otro detalle es que el vector inicial $x_0$ es elegido aleatoriamente. Si bien es posible en teoría que con un cierto $x_0$ el método no converja, en la práctica eso no sucede. En primer lugar, hay muy pocas probabilidades de tomarlo, y en segundo lugar, los errores de precisión juegan a nuestro favor, ya que con una mínima desviación el $x_0$ deja de ser problemático. \\

\todo[inline]{El último factor a tener en cuenta es la cantidad de iteraciones para la convergencia. Para que pasen los tests de la cátedra es necesario hacer más de 1200 iteraciones. Sin embargo, como veremos en la sección de experimentación, con una cantidad muchísimo menor de iteraciones la accuaracy no se modifica, por lo que vamos a tomar EQUIS iteraciones para optimizar el tiempo de ejecución.\\
 Criterio de convergencia, cant de iteraciones. IDEA: Experimento con un cierto set de datos (pequeño). dejar todo fijo e ir cambiando la cantidad de iteraciones y ver como varia el accuaracy y el tiempo.}





\section{K cross fold validation}

\todo[inline]{Que es?}

\todo[inline]{Por que nos sirve? que buscamos lograr? (evitar overfitting)}

\todo[inline]{Alguna imagen ilustrativa}

\todo[inline]{No se si vale la pena mencionar la implementacion, pensar que mas}




% -------------------------------------------------------
% Experimentación
% -------------------------------------------------------
\newpage
\section{Resultados finales y experimentación}

\subsection{Preliminares}

Dado que el objetivo general será tratar de estimar los mejores parámetros para nuestro sistema, haremos listado y una breve descripción de ellos.

\begin{itemize}
\item \textit{iters}: Cantidad de veces que itera el método de la potencia. Cuántas más iteraciones se hagan, mayor será la aproximación al autovalor real, pues en teoría el valor debería obtenerse en el límite.

\item \textit{k\_kNN}: Cantidad de vecinos a considerar en el método de los $k$ vecinos más cercanos (\textit{kNN}).

\item \textit{K\_kfold}: Cantidad de grupos (\textit{folds}) en el que vamos a dividir nuestro set de datos en los experimentos, para la aplicación de \textit{K-fold cross validation}

\item \textit{alpha ($\alpha$)}: Cantidad de dimensiones consideradas para el análisis de componentes principales (\textit{PSA}).

\end{itemize}

\todo[inline] {Explicar medidas que usamos, accuracy, quiza precision/recall}

\subsection{Método de la potencia}

Lo primero que nos gustaría ajustar será la cantidad de iteraciones necesarias para el método de la potencia, ya que es algo necesario para todas las mediciones que hagamos. Como adelantamos, para que pasen los tests de la cátedra son necesarias mas de 1000 iteraciones, sin embargo, veremos que con muchas menos iteraciones nuestro accuracy no se modifica. \\

El siguiente experimento fue realizado tomando 10000 muestras de entrenamiento, dejando completamente fijas todas las variables excepto la cantidad de iteraciones. Lo único que intentamos determinar es la cantidad de iteraciones para las cuales la accuracy llega a su máximo. \\

{\centering
    \includegraphics[scale=0.60]{informe/imagenes/potencia/accuracyPorIters.pdf} \\
    \captionof{figure}{Accuracy por cantidad de iteraciones. \\
    Todas las variables fijas \\ }
}
$ $\newline

Como podemos ver, la accuracy se mantiene constante, con la excepción de las cercanías de 10 iteraciones dónde aún son demasiado pocas. Consideramos que no es necesario tomar una cantidad de iteraciones demasiado alta, ya que con aproximadamente 50 parece ser más que suficiente. Es lógico pensar que a mayor cantidad de iteraciones, más tarda nuestro sistema en entrenar. Veamos si el incremento es significativo: \\

{\centering
    \includegraphics[scale=0.60]{informe/imagenes/potencia/tiempoPorIters.pdf} \\
    \captionof{figure}{Tiempo por cantidad de iteraciones. \\
    Todas las variables fijas \\ }
}
$ $\newline

La diferencia de tiempo es bastante notoria, sobre todo sabiendo que no estamos considerando la cantidad total de datos de entrenamiento. Dado que no encontramos diferencias de accuracy pero sí de tiempo, a partir de ahora fijaremos el parámetro de iteraciones en 50. \\


\todo[inline]{Graficar tiempos de knn para 10000 muestras, moviendo el k. Los datos ya estan tomados en knnMediciones, solo hay que extraerlos bien}


Nos gustaría analizar qué tan bien (o qué tan mal) se comporta kNN a medida que vamos variando el \textit{k}. \\

Dado que nuestros vectores viven en $\mathbb{R}^{784}$, esperamos que los resultados aplicando kNN (sin reducir dimensiones) sean malom por problemas con la \textit{maldición de la dimensión}. Además, no estamos considerando el espacio de muestras completo, por lo que esperamos que los resultados sean aún peores. \\

Sorprendentemente, lo que sucedió es todo lo contrario a lo que esperábamos, y con kNN obtuvimos (en promedio) precisiones de 0.95. \\

La mayoría de las clases se comportan de manera similar, salvo algunas excepciones que veremos más adelante. Veamos por ejemplo la clase del 0, que fue la \textit{mejor} clase.

{\centering
    \includegraphics[scale=0.55]{informe/imagenes/knn/precisionClase0.pdf} \\
    \captionof{figure}{Clasificación para clase 0, sólo kNN.\\Precision, Recall y F1, variando k.\\}
}
$ $\newline

Dado que la mayoría de los gráficos son similares, sólo graficaremos a continuación aquellas clases que más se diferencian.

{\centering
    \includegraphics[scale=0.70]{informe/imagenes/knn/precisionClase1257.pdf} \\
    \captionof{figure}{Clasificación para clases 1, 2, 5 y 7, sólo kNN\\Valores de Precision, Recall y F1, variando k.\\}
}
$ $\newline


\subsection{TODO:}

\todo[inline]{Experimentacion con knn variando los k-knn y los K-folds, para encontrar el mejor parametro posible}

\todo[inline]{Experimentacion de tiempo variando la cantidad de imagenes, no hace falta mil experimentos, un par de puntos clave, digamos 100, 1000, 2000, 5000, 10000, 20000, 40000}

\todo[inline]{Comparacion tiempos con, digamos, 10000 training entre knn a secas y con psa.}

\todo[inline]{Comparacion accuaracy y cosas con, digamos, 10000 training entre knn a secas y con psa.}

\todo[inline]{Podemos variar los K-kfolds si la base de entrenamiento es 'chica'}

\todo[inline]{Para entrenamientos con TODAS las imagenes y PSA, QUIZA nos convenga fijar los kfold en, nose, 5, y solo variar los demas parametros. Es RE costoso, (horas) calcular los K folds con un training grande (CON PSA).  Esta piola la idea de agustin de guardarnos en un txt la matriz de covarianza. \\
En limpio, la idea seria: Para 5 folds calcular y guardar las 5 matrices de covarianza, y despues experimentar variando los alphas y los k-knn. Esto seria 'barato' dentro de todo y podrian hacerce un par de graficos. Si hay tiempo variar el k-KFOLD pero lo dejaria para lo ultimo, mejor conseguir resultados concretos antes.}

\todo[inline]{Al final, mostrar resultados de Kaggle con y sin PSA, ir probando con un par de buenos parametros cuando los obtengamos}

\todo[inline]{Me parece muy piola de probar con manuscritos hechos por nosotros. Solo si sobra tiempo!}

% -------------------------------------------------------
% Conclusiones
% -------------------------------------------------------
\newpage
\section{Conclusión}

Concluimos que quedó re zarpado en piola.\\



ESTO DE ABAJO ESTÁ COMO COSA PREELIMINAL, PUEDEN CAMBIAR, BORRAR Y AGREGAR LO QUE QUIERAN\\


En este trabajo analizamos e implementamos una técnica de reonocimiento de dígitos. Utilizamos el método PCA para reducir la cantidad de variables de las muestras y agilizar la ejecución y la técnica kNN en el entrenamiento del programa. Luego corroboramos con K-fold los niveles de precision, recall y f1 obtenidos en dicho entrenamiento.\\

Estimamos parámetros de las técnicas utilizadas (PCA, kNN y Kfold) para buscar los mejores resultados y que consuman menor tiempo de cómputo.\\

Finalmente utilizamos la implementación realizada para participar en la competencia de Kaggle y evaluar la efectividad de nuestro programa.\\

Comprendimos lo fundamental que resulta el estudio de los elementos teóricos vistos en la materia (en particular todo lo relacionado a autovalores, autovectores y diagonalización de matrices, aplicado en PCA) y desarrollamos satisfactoriamente la técnica de reconocimiento de dígitos.

% -------------------------------------------------------
% Apéndice A
% -------------------------------------------------------
\newpage
\input{informe/Aappend.tex}

% -------------------------------------------------------
% Apéndice B
% -------------------------------------------------------
\newpage
\input{informe/Bappend.tex}

% -------------------------------------------------------
% Referencias
% -------------------------------------------------------
% \newpage
% \input{informe/ref.tex}


\newpage
\bibliographystyle{plain}
\bibliography{tp1}

\end{document}
